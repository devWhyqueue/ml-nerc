\section{Conclusion}
\label{sec:conclusion}

In this work, we presented a machine learning approach to Named Entity Recognition and Classification for drug names in biomedical text. We implemented a sequence tagging model using the B-I-O encoding scheme and compared two probabilistic classifiers: a Conditional Random Field and a Naïve Bayes model. Our experimental results demonstrated the effectiveness of the CRF for this task, achieving a macro-F1 score of approximately 72\% on the test set.

The feature engineering process played a crucial role in our system's performance. We developed a rich set of features capturing lexical, orthographic, and contextual information, as well as incorporating external domain knowledge through drug name lexicons. The ablation study highlighted the importance of lexicon features and contextual information, suggesting that domain knowledge and local context are particularly valuable for drug name recognition.

While our approach achieved competitive results, there are several avenues for future improvement. First, more sophisticated sequence models, such as neural networks with attention mechanisms or transformer-based models, could potentially capture more complex patterns in the data. Second, expanding the lexicons or developing methods to automatically extract potential drug names from unlabeled text could enhance the coverage of external knowledge. Third, incorporating additional domain-specific features, such as chemical structure information or semantic relationships between entities, might help resolve some of the challenging cases identified in our error analysis.

The task of drug name recognition remains important for biomedical information extraction, with applications in pharmacovigilance, drug safety monitoring, and literature mining. Our work contributes to this field by demonstrating the effectiveness of feature-based sequence modeling and highlighting the value of domain knowledge integration. The insights gained from our experiments and error analysis provide guidance for future research in this area, particularly for approaches that aim to balance performance with interpretability and domain adaptation~\cite{ddi-lessons}.
