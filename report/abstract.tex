\begin{abstract}
We present a machine learning approach to Named Entity Recognition and Classification (NERC) for drug names in biomedical text. The task, based on the DDI (Drug-Drug Interaction) corpus, involves identifying mentions of pharmacological substances and classifying them into four categories (drug, brand, group, drug\_n). We implement a sequence tagging model using the B-I-O encoding scheme and a Conditional Random Field (CRF) classifier~\cite{crf-tutorial} to jointly model token-level features and label transitions. A rich set of features is engineered, including lexical form, orthographic casing patterns, word suffixes, and contextual cues from neighboring tokens. We also integrate external domain knowledge through a lexicon of known drug names and categories to improve recognition of rare entities. The system is evaluated on the standard DDI corpus split into training, development, and test sets~\cite{ddi-corpus}. Experimental results show that our CRF-based NER model achieves high accuracy, substantially outperforming a baseline Naïve Bayes classifier. On the development set, the CRF attains a macro-averaged F1-score of $\sim$63\% using only local features, which further rises to $\sim$75\% with the incorporation of external drug name dictionaries. Final evaluation on the held-out test set yields a macro-F1 of approximately 72\%, approaching the performance of the best systems in the SemEval-2013 DDI challenge~\cite{semeval2013}. We provide a detailed analysis of feature impact, discuss common errors (especially for the challenging ``drug\_n'' class), and conclude with insights into improving NERC for biomedical text.
\end{abstract}
